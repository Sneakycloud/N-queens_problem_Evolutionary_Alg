% TARI29_Assignment_report_template.tex
% This is a template for the assignment reports
% TARI29 Artificial Intelligence, 7.5 credits, Winter 2024
% Original version by:  Vladimir Tarasov, 2024
% Revised by:           Alexandros Tzanetos, 2024 

% Use a modern class instead of an old article class
\documentclass{scrartcl}

\KOMAoptions{
    parskip=half,  % full, off
    fontsize=12pt, % base font size (10pt default)
    % headings=big,% small/normal/big headings (normal is default), 
    % paper=a5,    % paper format (a4 default) 
    pagesize=auto  % Use paper format for PDF too
}

% ---------------------- Report details --------------------- %
%\newcommand{\docauthor}{Name1, Name2, Name3}
\newcommand{\courseyear}{2024}
\newcommand{\coursename}{TARI29 Artificial Intelligence}
\newcommand{\coursecredits}{7.5 credits}
\newcommand{\fullcoursename}{\coursename, \coursecredits}
\newcommand{\termname}{Winter \courseyear}
\newcommand{\groupnumber}{\#}
\newcommand{\reportname}{Group \groupnumber\ Assignment Report}
\newcommand{\assignmentname}{Evolutionary Computation Assignment}
% ------------------ End of report details ------------------ %



% ---------- Setting up properties of the PDF file ---------- %
\usepackage{hyperref}
\hypersetup{
    colorlinks=true,
    allcolors=blue,
    pdftitle={\reportname},
    pdfauthor={Name1, Name2, Name3},
    pdfsubject={\coursename, \termname}
	% pdfpagemode=FullScreen,
	% pdfpagemode= UseOutlines % the default if present
}
\usepackage{hyphenat}
% ------------ End of properties of the PDF file ------------ %



% ----------- Setting up the headers and footers ------------ %
\usepackage{scrlayer-scrpage}
\pagestyle{scrheadings}
\KOMAoptions{% 
    headsepline,% line below the header
    plainheadsepline,% also on scrplain
}
\clearscrheadfoot % Erase all current configurations
% inner part of the header [titel_page]{other_pages}
\ihead[\coursename, \courseyear]{\coursename, \courseyear}
% outer part of the header [titel_page]{other_pages}
\ohead{\assignmentname}
% center part of the footer [titel_page]{other_pages}
\cfoot[\textup{Page \pagemark\ of \ztotpages}]{\textup{Page \pagemark\ of \ztotpages}}
% ------------- End of the headers and footers -------------- %



% ---------------- Setting up the title page ---------------- %
\title{\reportname}
\subtitle{An Evolutionary Algorithm for the XYZ problem}
%\author{\docauthor}
\author{Name1\and Name2\and Name3}
\date{\today}
% ------------------ End of the title page ------------------ %



% --------------- Packages and custom commands -------------- %
% Specify the input encoding of the source file as UTF8 
\usepackage[utf8]{inputenc}
% The T1 font encoding is an 8-bit encoding and fully supports words containing accented characters
\usepackage[T1]{fontenc}
% Load Latin modern fonts in outline format
\usepackage{lmodern}
% Load better typewriter font
\usepackage{inconsolata}
% For better hyphenation patterns
\usepackage[english]{babel}
% For improved micro-typography
\usepackage{microtype}
% Switch off extra space after a sentence
\frenchspacing
% To be able to iclude pictures
\usepackage{graphicx}
\usepackage{zref-totpages} % To get the total number of pages
\usepackage{mathtools,amsmath} % for advanced math formulas
 % To highlight source code
\usepackage{minted}
% Default options for all minted environments
\setminted{
           style=default, % for minted envronment
           autogobble, % automatically remove all common leading whitespace
           xleftmargin=10pt, % indentation to add (on the left) before the listing
           numbersep=6pt, % gap between numbers and start of line
           linenos, % enables line numbers
           mathescape % enables the usual math mode inside comments
}
\setmintedinline[c]{style=default} % for minted inlines
% Flexible handling of verbatim text
\usepackage{fancyvrb}
% For well-spaced lines and guidelines in tables
\usepackage{booktabs}
% To typeset algorithms or pseudocode in LaTeX 
\usepackage{algorithm}
\usepackage{algpseudocode}
% For plots and graphs
\usepackage{pgfplots}
\pgfplotsset{width=10cm,compat=1.9}

% To generate placeholder text - can be deleted when you have written real text
\usepackage{lipsum}
% ----------- End of packages and custom commands ----------- %


\begin{document}

\maketitle

% % It can be commented out if you do not want the table of contents
% \tableofcontents 


\section{Introduction}
\label{sec:intro}

\textcolor{red}{The recommended software for typesetting assignment reports is \LaTeX. It will allow you to prepare high-quality documents, especially in the area of Computer Science. This document can serve as a template for reports. Each section begins with brief instructions in red text. All the instructions in red, as well as the dummy text, should be removed in the final version to submit. The \LaTeX\ source of this file includes examples of using the most needed commands and environments. You can find plenty of other examples with explanations in many web forums and discussion groups on the Internet. The easiest way to edit your report is to use \url{https://www.overleaf.com/}. Overleaf does not require any setup on your computer, and it is free to create an account.}

\textcolor{red}{The book \textit{Writing for Computer Science} \cite{zobel2014writing} is a useful assistance on how to write properly and present your work when it comes to Computer Science topics. It is a strong recommendation to follow its guidelines and limit the usage of AI tools to generate text. Keep in mind that the examiner is an expert in Evolutionary Computation and therefore, any false information generated by an AI tool is easily notable. Such case may lead to failing the assignment.}

\textcolor{red}{The introduction should briefly introduce the assignment and its purpose.}

\lipsum[4]


\section{Problem}
\label{sec:problem_description}

\textcolor{red}{The second section should present the problem you tackle using your evolutionary approach. Overall, this section should include:}

{\color{red}
\begin{itemize}
    \item The mathematical formulation considered in your study. Some problems have a clear mathematical model (e.g., Travelling Salesman Problem), while others do not (e.g., $n$-Queens). Based on the problem you chose, search the literature and find a proper way to present the problem.
    \item One paragraph that briefly presents at least 3 published academic works where any evolutionary approach is used to solve the problem. It would be wise to cite here works that influenced your algorithm. This practice saves you time from looking for additional academic resources. You can find more information about reading and searching in the literature in \cite{zobel2014reading}.
    \item The motivation behind the evolutionary approach you decided to develop. A good practice would be to align the motivation with some literature gap found in the academic works you presented above. However, this is not mandatory. You can motivate your selection on the characteristics of the algorithm making it proper for the problem.
\end{itemize}
}

\textcolor{red}{\textbf{Note:} Change the section's title to match the name of the problem you chose for your assignment.}

\lipsum[2]


\section{Algorithm}
\label{sec:algorithm}

\textcolor{red}{The third section should present the evolutionary approach you developed. You can divide this section into subsection. In any case, you should mention the following details:}

\textcolor{red}{\textbf{Evolutionary approach.} Clearly describe the algorithm you developed. You should clearly explain the evolutionary operators you used and what modifications you did to match the problem. It is extremely important to present also a pseudocode of your algorithm. An example is given in \ref{alg:pseudocode_example}, below. For more insight into presentation of algorithms, you can advise \cite{zobel2014algorithms}.}

{
\color{red}To typeset pseudocode in \LaTeX\ you can use one of the following options:
\begin{itemize}
    \item Choose ONE of the (\texttt{algpseudocode} OR \texttt{algcompatible} OR \texttt{algorithmic}) packages to typeset algorithm bodies, and the algorithm package for captioning the algorithm.
    \item The \texttt{algorithm2e} package.
\end{itemize}
You can find more information here: \url{https://www.overleaf.com/learn/latex/Algorithms}
}

\begin{algorithm}
\caption{Example of an algorithm's pseudocode}\label{alg:pseudocode_example}
\begin{algorithmic}
\Require $n \geq 0$
\Ensure $y = x^n$
\State $y \gets 1$
\State $X \gets x$
\State $N \gets n$
\While{$N \neq 0$}
\If{$N$ is even}
    \State $X \gets X \times X$
    \State $N \gets \frac{N}{2}$  \Comment{This is a comment}
\ElsIf{$N$ is odd}
    \State $y \gets y \times X$
    \State $N \gets N - 1$
\EndIf
\EndWhile
\end{algorithmic}
\end{algorithm}

\textcolor{red}{\textbf{Solution representation.} Clearly describe the solution representation you used. You can use figures to improve the comprehensibility of this part.}

\textcolor{red}{\textbf{Fitness function.} It is also very important to mention the fitness function you used. In many cases, the objective function of the problem is not the same as the fitness function used in an evolutionary algorithm. An example, following the principles of \cite{zobel2014mathematics}, is given below.}

\begin{equation}
    F = \sum_{i=1}^d x_i^2 
\end{equation}
where $x_i$ is the $i$-th gene (i.e., decision variable) in the solution and $d$ corresponds to the number of decision variables in the problem.

\textcolor{red}{\textbf{Note:} Change the section's title to match the name of the algorithm you developed for your assignment.}

\lipsum[3]


\section{Experimental part}
\label{sec:experimentation}

{\color{red}
This section describes the setup of experiments \cite{zobel2014experimentation}:

\begin{itemize}
    \item Provide the details of the hardware and software that you used.
    \item Describe the steps you carried out during your experiments.
    \item Detail the data you used for the evaluation of your algorithm.
\end{itemize}
}

\textcolor{black}{The algorithms and the hardware used for the experiment. The experiments were performed on a AMD Ryzen 7 5700X 8-Core Processor. With a clock frequency of 3401 Mhz. The algorithm that was outlined in the algorithm section is implemented in python at \url{https://github.com/Sneakycloud/N-queens_problem_Evolutionary_Alg/tree/main}. The repository also includes a "Base algorithm" folder in a simple evolutionary algorithm is included.}

\textcolor{black}{\newpage The simple algorithm is designed in a much more simple manner. }
\begin{itemize}
	\item Representation: It uses the same ordered list representation as the main algorithm. However, it does not guarantee that all column indexes are unique throughout the algorithm.
	\item Population initialization:  Creates a list with values 0 to n-1. Without randomizing order.
	\item Recombination: Selects a start and end index within the length of the first parent. The selected interval of the first parent is combined the missing indexes from the second parent to create a child.
	\item Mutation: Randomly adds or subtracts one from a random index as long as the result is within bounds.
	\item Selection: Randomly picks boards from the old generation until the new generation is filled to the specified limit.
	\item fitness: In addition to calculating the diagonals, the base algorithm also counts the number of duplicate column indexes and adds it to the resulting sum of conflicts. 
\end{itemize}

\textcolor{black}{The method of the experiment consists of running each algorithm at different N-lengths of the board for 100 iterations. The algorithms are tested at the following sizes of N: 4, 8, 10, 20, 30, 40, 50. The other parameters such as generation size, amount of children created each generation and mutation rate are tuned to find a local optimum parameters for the algorithms at that board size of N.  The appendix will include some examples of parameters and their results.}

\textcolor{black}{Instructions to run the algorithms. To run the main algorithm download it from the \href{https://github.com/Sneakycloud/N-queens\_problem\_Evolutionary\_Alg/tree/main}{github} and then run "Python main.py" in the directory. All parameters can be adjusted in the function call at the bottom of the file. To run the base algorithm go into the "Base algorithm" folder and run "python base.py". Its parameters can also be adjusted at the bottom of the file.}

\textcolor{black}{The measured parameters is the time and generations until the algorithm finds an solution. The method used to measure time is the python function "time.process\_time()" which only accounts for the time the process is active and will not count thread sleep time. For the counting the generations a simple counter which ticks up at the end of each while loop is sufficient.}

\section{Results and Analysis}
\label{sec:results-analysis}

\textcolor{red}{This section should present the obtained results and provide an insightful analysis of them. You can present the results using graphs, tables, or any other visualization method suits your purpose. Do not forget to include proper captions \cite{zobel2014graphs} in any of these illustration methods you use. You do not need to provide any execution details as they are already presented in Sec.~\ref{sec:experimentation}.}

\textcolor{red}{A good practise would be to compare your algorithm with a simpler approach, such as (a) a naive method, (b) a Hill Climbing approach, or (c) a simple evolutionary algorithm. In the third case, you can use the simpler version of the algorithm you developed, i.e., the original algorithm without your modifications. In that case, you should briefly describe the comparing method(s) in Sec.~\ref{sec:experimentation}. Alternatively, you can use some reference results derived from the repositories you found some benchmark instances.}

\textcolor{red}{To display tables, the \texttt{booktabs} package might be useful. For example, Table~\ref{tab:results_example} shows how you should increase the  size of $n$, when running your code. You can advice \cite{zobel2014graphs} to see a few examples of proper tables.}

\begin{table}[h]
\centering
\caption{Example of comparison the developed algorithm's results with the best ones from a repository.}
\label{tab:results_example}
	\begin{tabular}{lrrr}
	\toprule
	\textbf{Instance} & \textbf{Optimum (Repository xyz)} & \textbf{EA} & \textbf{time (s)}\\
	\midrule
	st70       &  678.597      & 677.109	& 0.67\\
	ei176      &  545.387      & 544.369	& 1.16\\
	kroA100    &  21285.443    & 21285.443	& 1.69\\
	rd100      &  7910.396     & 7910.396	& 2.14\\
	Pr136      &  96772        & 96770.924	& 7.11\\
	Pr144      &  58537        & 58535.221	& 7.97\\
	a280       &  2856.769     & 2856.769	& 33.47\\
	\bottomrule
	\end{tabular}
\end{table}

\textcolor{red}{You can use different illustration methods to present different aspects of your analysis. Figure~\ref{fig:plot_example} gives an example using the \href{https://www.overleaf.com/learn/latex/Pgfplots_package}{\texttt{pgfplots}} package.}

\begin{figure}[h]
\centering
\begin{tikzpicture}
\begin{axis}[
    % title={Example of convergence analysis},
    xlabel={Generation},
    ylabel={Fitness function value},
    xmin=0, xmax=20,
    ymin=290, ymax=450,
    xtick={0,5,10,15,20},
    ytick={290,300,350,400,450},
    legend pos=north east,
    ymajorgrids=true,
    grid style=dashed,
]

\addplot[
    color=blue,
    mark=square,
    ]
    coordinates {
    (0,420)(4,411)(7,387)(10,382)(14,364)(18,360)(20,358)
    };
    \addlegendentry{EA1}

\addplot[
    color=red,
    mark=triangle,
    ]
    coordinates {
    (0,422)(4,373)(9,362)(12,312)(18,311)(20,309)
    };
    \addlegendentry{EA2}

\addplot[
    color=violet,
    mark=diamond,
    ]
    coordinates {
    (0,448)(6,401)(8,349)(11,325)(14,299)(20,298)
    };
    \addlegendentry{EA3}
  
\end{axis}
\end{tikzpicture}
\caption{Example of convergence analysis.}
\label{fig:plot_example}
\end{figure}

\lipsum[4]

\section{Conclusions}
\label{sec:conclusions}

\textcolor{red}{In this section you should provide a concise summary of what has been done, the obtained results and some recommendations on how this study could be extended.}

\lipsum[4]

\bibliographystyle{ieeetr}
\bibliography{references}

\end{document}
